\section{Introduction}
\subsection{Context}
We are in 2050

\subsection{Case Study}
The study will simulate and SMR installed in a prevalently residential area.
The reactor will be coupled with a secondary system and the electricity production should be able to cover the entire demand of the area.
The model also takes into account the share of renewable energy sources in the area.

\subsubsection{Reactor}
Our plant consists in an SMR installed in the Joint Research Center (JRC) in Ispra, Northen Italy.
The JRC is a research facility of the European Commission, located on the shores of Lake Maggiore.
The facility is home to the ESSOR reactor, currently facing decommissioning.
The facility has been chosen since it rapresents an areleady nuclearized site which suggests the possibility of a future nuclear plant.
Moreover the installation of the SMR could take place, technically, withing the ESSOR reactor considering that the containment building 
was oversized due to the particular use case of the ESSOR reactor in order to accomodate not only the reactor iteself but also several laboratories.

\subsubsection{Secondary Plant}

\subsection{Objectives}
The objective of this work is to determine the feasibility of this installation both technically and economically.
On the techincal side of things we foucs on the capability of the system to cover the energy deamnd and to adapt to transient conditions.
The economical side of things will assest the minimum total plant cost that would allow for a positive NPV.  